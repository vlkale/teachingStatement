\documentclass{article}

%\documentclass{beamer}
%\usepackage{inputenc}
%\usetheme{Luebeck}
%\usefonttheme{serif}
%\usecolortheme{rose}
%\useoutertheme{infolines}

\usepackage{graphics}
\usepackage{ulem}
\usepackage{color}
\usepackage{xspace}
\usepackage{amssymb}
\usepackage{amsmath}
\usepackage{listings}

\definecolor{codegreen}{rgb}{0,0.6,0}
\definecolor{codegray}{rgb}{0.5,0.5,0.5}
\definecolor{codepurple}{rgb}{0.58,0,0.82}
\definecolor{backcolour}{rgb}{0.95,0.95,0.92}

\lstset{
        language=C++,
        basicstyle=\small\ttfamily, % Standardschrift %                                                                                        
                                    % % TODO: consider making tiny to make                                                                     
                                     % % things f                                                                                              
         numbers=none,               % Ort der Zeilennummern                                                                                   
         numberstyle=\footnotesize\ttfamily,          % Stil der Zeilennummern                                                                 
         %stepnumber=2,               % Abstand zwischen den Zeilennummern                                                                     
         numbersep=6pt,              % Abstand der Nummern zum Text                                                                            
         tabsize=2,                  % Groesse von Tabs                                                                                        
         extendedchars=true,         %                                                                                
         breaklines=true,            % Zeilen werden Umgebrochen                           
         frame=single,
         keywordstyle=[1]\textbf,    % Stil der Keywords                          
         stringstyle=\ttfamily,
         showspaces=false,           % Leerzeichen anzeigen ? 
         showtabs=false,             % Tabs anzeigen ?                         
         showstringspaces=false,     % Leerzeichen in Strings anzeigen ?
         commentstyle=\color{codegreen},
         keywordstyle=\color{codepurple},
         numberstyle=\tiny\color{codegray},
         keywordstyle=\color{blue},
         stringstyle=\color{red},
         morecomment=[l][\color{magenta}]{\#},
         captionpos=b,
         numbers=left,
         numbersep=5pt,
         showspaces=false,
}

%\usepackage{beamerthemesplit}

\usepackage[fencedCode,inlineFootnotes,citations,definitionLists,hashEnumerators,smartEllipses,hybrid]{markdown}

\markdownSetup{rendererPrototypes={
 link = {\href{#2}{#1}},
 headingThree = {\begin{frame}\frametitle{#1}},
 headingFour = {\begin{block}{#1}},
 horizontalRule = {\end{block}}
}}

%\title{h}
%\author{h}
%\date{h}

\newcommand\NameOfUniversity[1]{Rutgers University}


\begin{document}

%\documentclass{article}

%\documentclass{beamer}
%\usepackage{inputenc}
%\usetheme{Luebeck}
%\usefonttheme{serif}
%\usecolortheme{rose}
%\useoutertheme{infolines}

\usepackage{graphics}
\usepackage{biblatex}
\usepackage{ulem}
\usepackage{color}
\usepackage{xspace}
\usepackage{amssymb}
\usepackage{amsmath}
\usepackage{listings}

\definecolor{codegreen}{rgb}{0,0.6,0}
\definecolor{codegray}{rgb}{0.5,0.5,0.5}
\definecolor{codepurple}{rgb}{0.58,0,0.82}
\definecolor{backcolour}{rgb}{0.95,0.95,0.92}

\lstset{
        language=C++,
        basicstyle=\small\ttfamily, % Standardschrift %                                                                                        
                                    % % TODO: consider making tiny to make                                                                     
                                     % % things f                                                                                              
         numbers=none,               % Ort der Zeilennummern                                                                                   
         numberstyle=\footnotesize\ttfamily,          % Stil der Zeilennummern                                                                 
         %stepnumber=2,               % Abstand zwischen den Zeilennummern                                                                     
         numbersep=6pt,              % Abstand der Nummern zum Text                                                                            
         tabsize=2,                  % Groesse von Tabs                                                                                        
         extendedchars=true,         %                                                                                                         
         breaklines=true,            % Zeilen werden Umgebrochen                                                                               
         frame=single,
         keywordstyle=[1]\textbf,    % Stil der Keywords                                                                                       
         stringstyle=\ttfamily,
         showspaces=false,           % Leerzeichen anzeigen ?                                                                                  
         showtabs=false,             % Tabs anzeigen ?                                                                                         
         showstringspaces=false,     % Leerzeichen in Strings anzeigen ?                                                                       
         commentstyle=\color{codegreen},
         keywordstyle=\color{codepurple},
         numberstyle=\tiny\color{codegray},
         keywordstyle=\color{blue},
         stringstyle=\color{red},
         morecomment=[l][\color{magenta}]{\#},
         captionpos=b,
         numbers=left,
         numbersep=5pt,
         showspaces=false,
}

%\usepackage{beamerthemesplit}

\usepackage[fencedCode,inlineFootnotes,citations,definitionLists,hashEnumerators,smartEllipses,hybrid]{markdown}

\markdownSetup{rendererPrototypes={
 link = {\href{#2}{#1}},
 headingThree = {\begin{frame}\frametitle{#1}},
 headingFour = {\begin{block}{#1}},
 horizontalRule = {\end{block}}
}}

%\title{h}
%\author{h}
%\date{h}

\begin{document}

\documentclass{article}

%\documentclass{beamer}
%\usepackage{inputenc}
%\usetheme{Luebeck}
%\usefonttheme{serif}
%\usecolortheme{rose}
%\useoutertheme{infolines}

\usepackage{graphics}
\usepackage{biblatex}
\usepackage{ulem}
\usepackage{color}
\usepackage{xspace}
\usepackage{amssymb}
\usepackage{amsmath}
\usepackage{listings}

\definecolor{codegreen}{rgb}{0,0.6,0}
\definecolor{codegray}{rgb}{0.5,0.5,0.5}
\definecolor{codepurple}{rgb}{0.58,0,0.82}
\definecolor{backcolour}{rgb}{0.95,0.95,0.92}

\lstset{
        language=C++,
        basicstyle=\small\ttfamily, % Standardschrift %                                                                                        
                                    % % TODO: consider making tiny to make                                                                     
                                     % % things f                                                                                              
         numbers=none,               % Ort der Zeilennummern                                                                                   
         numberstyle=\footnotesize\ttfamily,          % Stil der Zeilennummern                                                                 
         %stepnumber=2,               % Abstand zwischen den Zeilennummern                                                                     
         numbersep=6pt,              % Abstand der Nummern zum Text                                                                            
         tabsize=2,                  % Groesse von Tabs                                                                                        
         extendedchars=true,         %                                                                                                         
         breaklines=true,            % Zeilen werden Umgebrochen                                                                               
         frame=single,
         keywordstyle=[1]\textbf,    % Stil der Keywords                                                                                       
         stringstyle=\ttfamily,
         showspaces=false,           % Leerzeichen anzeigen ?                                                                                  
         showtabs=false,             % Tabs anzeigen ?                                                                                         
         showstringspaces=false,     % Leerzeichen in Strings anzeigen ?                                                                       
         commentstyle=\color{codegreen},
         keywordstyle=\color{codepurple},
         numberstyle=\tiny\color{codegray},
         keywordstyle=\color{blue},
         stringstyle=\color{red},
         morecomment=[l][\color{magenta}]{\#},
         captionpos=b,
         numbers=left,
         numbersep=5pt,
         showspaces=false,
}

%\usepackage{beamerthemesplit}

\usepackage[fencedCode,inlineFootnotes,citations,definitionLists,hashEnumerators,smartEllipses,hybrid]{markdown}

\markdownSetup{rendererPrototypes={
 link = {\href{#2}{#1}},
 headingThree = {\begin{frame}\frametitle{#1}},
 headingFour = {\begin{block}{#1}},
 horizontalRule = {\end{block}}
}}

%\title{h}
%\author{h}
%\date{h}

\begin{document}

\documentclass{article}

%\documentclass{beamer}
%\usepackage{inputenc}
%\usetheme{Luebeck}
%\usefonttheme{serif}
%\usecolortheme{rose}
%\useoutertheme{infolines}

\usepackage{graphics}
\usepackage{biblatex}
\usepackage{ulem}
\usepackage{color}
\usepackage{xspace}
\usepackage{amssymb}
\usepackage{amsmath}
\usepackage{listings}

\definecolor{codegreen}{rgb}{0,0.6,0}
\definecolor{codegray}{rgb}{0.5,0.5,0.5}
\definecolor{codepurple}{rgb}{0.58,0,0.82}
\definecolor{backcolour}{rgb}{0.95,0.95,0.92}

\lstset{
        language=C++,
        basicstyle=\small\ttfamily, % Standardschrift %                                                                                        
                                    % % TODO: consider making tiny to make                                                                     
                                     % % things f                                                                                              
         numbers=none,               % Ort der Zeilennummern                                                                                   
         numberstyle=\footnotesize\ttfamily,          % Stil der Zeilennummern                                                                 
         %stepnumber=2,               % Abstand zwischen den Zeilennummern                                                                     
         numbersep=6pt,              % Abstand der Nummern zum Text                                                                            
         tabsize=2,                  % Groesse von Tabs                                                                                        
         extendedchars=true,         %                                                                                                         
         breaklines=true,            % Zeilen werden Umgebrochen                                                                               
         frame=single,
         keywordstyle=[1]\textbf,    % Stil der Keywords                                                                                       
         stringstyle=\ttfamily,
         showspaces=false,           % Leerzeichen anzeigen ?                                                                                  
         showtabs=false,             % Tabs anzeigen ?                                                                                         
         showstringspaces=false,     % Leerzeichen in Strings anzeigen ?                                                                       
         commentstyle=\color{codegreen},
         keywordstyle=\color{codepurple},
         numberstyle=\tiny\color{codegray},
         keywordstyle=\color{blue},
         stringstyle=\color{red},
         morecomment=[l][\color{magenta}]{\#},
         captionpos=b,
         numbers=left,
         numbersep=5pt,
         showspaces=false,
}

%\usepackage{beamerthemesplit}

\usepackage[fencedCode,inlineFootnotes,citations,definitionLists,hashEnumerators,smartEllipses,hybrid]{markdown}

\markdownSetup{rendererPrototypes={
 link = {\href{#2}{#1}},
 headingThree = {\begin{frame}\frametitle{#1}},
 headingFour = {\begin{block}{#1}},
 horizontalRule = {\end{block}}
}}

%\title{h}
%\author{h}
%\date{h}

\begin{document}

\input{teachingStatementWWU.md}
%\markdownInput{./teachingStatementWWU.md}

\end{document}

%\markdownInput{./teachingStatementWWU.md}

\end{document}

%\markdownInput{./teachingStatementWWU.md}

\end{document}

\markdownInput{./teachingStatement_NameOfUniversity_NameOfPos.md}


\bibliographystyle{alpha}
\bibliography{bibliography}
%\thebibliography

\end{document}
